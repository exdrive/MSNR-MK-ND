\documentclass{beamer}

\usetheme{Copenhagen}
\useoutertheme{shadow}
\usecolortheme{orchid}
\setbeamertemplate{navigation symbols}{}

\usepackage[utf8]{inputenc}
\usepackage[serbian]{babel}
\usepackage{graphicx}
\usepackage{color}
\usepackage{url}
\usepackage{graphicx}
\usepackage{caption}
\captionsetup[]{name=Karikatura}

\title{Pravne i etičke obaveze u IT svetu}
\institute{Matematički fakultet}
\author{Nikola Dokmanović, Miloš Krsmanović}
\date{\today}   %proveriti da datum bude dobar!!!

%PREZENTACIJA_STRUKTURA: NASLOVNI SLAJD, 6(7) SLAJDOVA, POZDRAVNI SLAJD, LITERATURA.

\begin{document}

%I
\frame{
	\titlepage
}

% II
\section{Uvod}
\subsection{Ukratko o svemu}
\frame{
	\frametitle{Ukratko o svemu}
	\begin{itemize}
		\item Potreba za etičkim smernicama\pause % O ovome ukratko pricam ja
		\item Neophodnost etičkog kodeksa. \pause % O ovome ukratko pricam ja
	
		\item ??? 	   \pause %O ovome ukratko pricas ti :D
		\item ??? 			  %O ovome ukratko pricas ti :D
	\end{itemize}
}
%III
\section{Etičke obaveze i pitanja u IT svetu}
\subsection{Etika}
\frame{
\frametitle{Šta je etika?}
	\begin{itemize}
		\item[•] Deo filozofije koji proučava moralne vrednost \pause
		\item[•] "Poznavanje razlike između onoga što imamo pravo da radimo i onoga što bismo trebali da
		radimo."] \pause
		\item[•] \textbf{Standard koji štiti privatnost informacija.}
	\end{itemize}
}

%Pozdravni slajd
\frame{
		\begin{itemize}
			%\vspace{1cm}
			\item[] \center{{\LARGE Hvala Vam na pažnji!}} \pause
			\vspace{0.5cm}
			\item[] \center{{\Large Pitanja?}}
		\end{itemize} 	
}
%Poslednji slajd - Literatura
\subsection{Literatura}
\frame{
\frametitle{Literatura}
		\begin{itemize}
			\item Prvi izvor
			\item Drugi izvor
			\item Treci izvor	
		\end{itemize} 	
}
\end{document}
