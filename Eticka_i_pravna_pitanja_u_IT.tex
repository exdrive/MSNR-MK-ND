%			Seminarski rad u okviru kursa Metodologija strucnog i naucnog rada 
\documentclass[a4paper]{article}
\usepackage{color}
\usepackage{url}
\usepackage[utf8]{inputenc}
\usepackage{multirow}
\usepackage[table,xcdraw]{xcolor}
\usepackage{graphicx}
\usepackage{adjustbox}
\usepackage[english,serbian]{babel}
\usepackage[unicode]{hyperref}
\hypersetup{colorlinks,citecolor=green,filecolor=green,linkcolor=blue,urlcolor=blue}
\newcommand\tab[1][1cm]{\hspace*{#1}}  
\newtheorem{primer}{Primer}[section]

\def\d{{\fontencoding{T1}\selectfont\dj}}
\def\D{{\fontencoding{T1}\selectfont\DJ}}

\begin{document}

\title{ Pravna i etička pitanja u IT  \\ \small{Seminarski rad u okviru kursa\\Metodologija stručnog i naučnog rada\\ Matematički fakultet}}

\author{Nikola Dokmanović, Miloš Krsmanović\\mi15117@alas.matf.bg.ac.rs, mi15263@alas.matf.bg.ac.rs}
\date{\today}
\maketitle 

\abstract{ Šta je etika? Zašto je ona neophodna u svetu danas? Koliku ulogu igraju moralna pitanja u svetu informacionih tehonologija? Ko je odgovoran za etiku u pogledu, recimo, izrade softvera koji se koristi za prenos podataka o pacijentima između medicinskih ustanova? Ovo su samo neka od pitanja kojima ćemo se pozabaviti i podstaći čitaoca na razmišljanje. 
Cilj ovog rada nije samo davanje odgovora i šturih informacija već iznošenje činjenica i naglašavnje tendencija u pravnim, etičkim i moralnim pitanjima.

\tableofcontents

\section{Uvod}
\label{sec:Uvod}
Nedovoljno upućeni ljudi često su mišljenja da u svetu informacionih tehnologija sve funkcioniše bez kolizija i nesuglasica. Kao i svaka druga grana privrede, tako se i IT suočavaju sa mnogim važnim pitanjima od kojih je jedno od značajnijih pitanje etike i morala. O etičkim pitanjim u svetu informacionih tehonologija nije lako govoriti, pa ipak, etičko ponašanje se očekuje od profesionalaca u gotovo svim granama industrije. Potrebno je, pre svega, definisati koje su to grane u IT-u koje zahtevaju bavljenje etikom. Skup svih grana informacionih tehnologija je znatno veći od drugih grana privrede i stoga je teško odrediti u  kojim granama je potrebno poštovati etički kodeks.

Kada je, na primer, život jedne osobe pogođen aspektima određenog softvera, pitanje sa kojima se IT zajednica suočava danas je da li projektanti i analitičari koji su uključeni u razvoj softvera ili programeri treba da osećaju potrebu za etičkim kodeksom.
Pravni odgovor na ovo pitanje ne postoji, jer kako se svet tehnologije razvija i napreduje veoma brzo, izrazito je teško sprovesti zakone i doneti ispravne odluke po pitanju morala i etike u svetu informaiconih tehnologija. Bez obzira na to, jasno da stanje etike u IT-u treba da se poboljša.

\section{Šta je etika?}

Formalno, etika bi se mogla okarakterisati kao deo filozofije koji pro\-u\-ča\-va i procenjuje moralne vrednosti. Ona predstavlja kodeks ponašanja i profesionalne organizacije koji poštuju (prate) institucije, ustanove i stručnjaci u svim zanimanjima kako bi se osigurala privatnost i poverenje korisnika. Etika ne uklanja odgovornost sa pojedinca i prenosi je na ustanovu, naprotiv, za etiku je neophodno razumevanje koje dolazi iznutra. Da bi čovek kao praktično biće usvojio moralne norme i po njima se ponašao, da bi formirao vrednosno-normativni odnos prema sebi, ali i prema drugim ljudima, mora da donese odgovarajući moralni sud. Moralni sud je sud o vlastitom ponašanju, ponašanju drugih ljudi i drugih društvenih grupa.\cite{Tavani}\cite{Illinois}
Često se navodi da je etika \emph{“poznavanje razlike između onoga što imamo pravo da radimo i onoga što bismo trebali da radimo”.}\footnote{Poter Stjuart, sudija Vrhovnog suda SAD od 1958-1981}

Najjednostavniji primer može biti poznata relacija lekar-pacijent gde je poverljivost podrazumevani standard koji štiti privatnost medicinskih informacija. U tom odnosu, diskrecija i poštovanje etike je nešto što se smatra prirodnim. Ipak, često je upravo pitanje etike skrajnuto i na njega se ne obraća dovoljno pažnje.

\subsection{Pojava etike u svetu informacionih tehnologija}

Razvoj kompjuterske tehnologije koji je doneo nebrojeno mnogo dobrog, stvorio je i brojne mogućnosti zloupotrebe. To je rezultovalo brojnim socijalnim problemima koji su stvorili nove moralne dileme. Upravo  etika pokušava da bude ključ njihovih rešavanja. U razvoju tehnološke etike mora se uzeti u obzir da ona nije samo filozofska disciplina, već važna praktična grana koja treba da reši brojna pitanja koja postavljaju sociolozi, političari, pravnici, kompjuterski tehničari i slični profesionalci\cite{Illinois}.

\subsection{Etička pitanja za IT stručnjake}

Lekari, advokati i drugi profesionalci čiji posao utiče na živote drugih ljudi obično pohađaju, kao deo njihove formalne obuke, kurseve koji se bave etičkim pitanjima u njihovoj profesiji. To nije slučaj sa IT stru\-čnja\-ci\-ma, a upravo oni često imaju pristup tajnim podacima i saznanjima o pojedincima i kompanijama. To je \emph{moć} koja se može zloupotrebiti, namerno ili nenamerno. Iako ne postoje standardizovane obuke za IT stručnjake, sve više kompanija koje proizvode softver počinje da obraća pažnju na etiku.

Profesionalci u mnogim strukama se često susreću sa pitanjima etike a da toga nisu ni svesni. Kao primer možemo navesti sledeću situaciju: Prelistavanjem slučajnih dokumenata otkrili ste važne službene tajne. Šta ako napustite tu kompaniju i zaposlite se kod konkurenata? Da li je pogrešno koristiti to znanje na novom poslu? Da li bi još gore bilo da ta dokumenta odštampate i ponesete ih sa sobom?
Moguć je, recimo, i ovakav scenario: Dokumenti koje ste čitali potvrđuju da kompanija krši vladine propise ili zakone. Da li će u vama prevladati moralna odgovornost da te nepravilnosti prijavite ili etička obaveza da poštujete privatnost svog poslodavca.\cite{Schneider}\cite{Reynolds}
Informatičkim stručnjacima trebalo bi omogućiti jasne smernice koje bi im pomogle u rešavanju etičkih pitanja.

\subsection{Potreba za etičkim smernicama}

Za razliku od starijih, davno ustanovljenih profesija kao što su medicina i pravo, većina moralnih i etičkih pitanja sa kojima se susreću stručnjaci u IT-u ne nalaze se ni u jednom zakonu ili propisu. Uzevši u obzir činjenicu da informacioni sistemi predstavljaju relativno novu tvorevinu i ukoliko se objektivno razmotri situacija, nije postojalo dovoljno vremena da se svi stavovi usaglase, te ne postoje velike zamerke na to što ujedinjeni etički kodeks još uvek ne postoji. 
Međutim, pitanje etičkog ponašanja u informacionim tehnologijama počinje da se rešava. Napor pojedinih udruženja profesionalaca, kao što je ACM\footnote{ACM, Američka Asocijacija za kompjuterske sisteme osnovana 1947. godine} da razvijaju svoje etičke kodekse profesionalnog ponašanja, koji mogu poslužiti kao smernica za pojedince i druge organizacije samo je korak napred u smeru donošenja jedinstvenog kodeksa jer njihovi stavovi gotovo ni u čemu nisu kontradiktorni.
\\
Iako jedinstven etički kodeks nije ostvaren, on bi trebalo da ima prednost nad postojećim u ostvarivanju sledećih ciljeva:
\\
\begin{itemize}
\item[•] \textbf{Inspiracija} - Kodeks bi trebao da motiviše informatičare da se ponašaju više u skladu sa etikom.
\item[•] \textbf{Disciplinovanost} - Kodeks bi trebao da doprinese uspostavljanju pravila među informatičarima koja će se poštovati.
\item[•] \textbf{Informisanost} - Kodeks bi trebao da jasno da do znanja svim mogućim poslodavcima i korisnicima usluga informatičara šta bi trebalo i šta bi mogli da očekuju.
\end{itemize}

%Navedeni ciljevi treba da predstavljaju motivaciju za izrađivanje detaljnog etičkog i pravnog kodeksa koji %bi se poštovao u svetu informacionih tehnologija.
Potreba za etičkim smernicama je utoliko veća zbog tendencije rasta etičkih dilema (i konflikata) u državnim ustanovama gde su sprovođena istraživanja na ovu temu.
U tabeli isprod prikazani su rezultati jednog od istraživanja koje je sprovođeno tri godine za redom.
\footnote{Napomena: Anketirani su zaposleni u preduzećima u Sjedinjenim Američkim Državama}

\begin{table}[h!]
%\centering
\begin{adjustbox}{width=\textwidth}
\begin{tabular}{|l|l|c|c|c|}
\hline
\multicolumn{2}{|c|}{\multirow{2}{*}{\textbf{Grupa zaposlenih koji su:}}}         & \multicolumn{3}{c|}{\textbf{Godina anketiranja}} \\ \cline{3-5} 
\multicolumn{2}{|c|}{}  & \textbf{2007.}    & \textbf{2008.}   & \textbf{2009.}   \\ \hline
\multicolumn{2}{|l|}{\small{Prisustvovali kršenju etičkih pravila}}      & 45\%  & 49\% & 63\%  \\ \hline
\multicolumn{2}{|l|}{\small{Bili primorani na kršenje moralnih pravila}} & 58\%  & 63\% & 65\%  \\ \hline
\multicolumn{2}{|l|}{\small{Primetili manjak etike u radnom okruženju}}  & 39\%  & 35\% & 42\%  \\ \hline
\end{tabular} 
\end{adjustbox} 
\caption{Rezultati ankete sprovedene u državnim preduzećima}
\label{}
\end{table} 

\section{Osnovna pravna pitanja vezana za IT}

Kako se tehnologija razvija potrebno je da je prate zakoni koji obez\-be\-đu\-ju korišćenje tehnologije. Razvijanjem tehnologije konstantno je potrebno unapređivati zakone vezane za informacijone tehnologije, ali zbog jako brze evolucije zakon jednostavno ne stiže da se unapredi dovoljnom brzinom.

Zakoni koji danas obezbeđuju trgovinu, privatnost podataka i intelektualnu svojinu na primer, pisani su za vremena kada se trgovalo sa listom papira, kada se komuniciralo preko telegrafa i kada su se dokumenti kucali na kucaćim mašinama.\footnote{Univerzitet Princeton, Legal Issues and Information Security} Sve ovo se dramatično razvilo tokom godina jačanjem informacionih tehnologija.

Obradićemo tri oblasti gde konstantna promena informacionih tehnologija drastično utiče na zakon:

\begin{itemize}
\item{Elektronska trgovina}
\item{Zaštita privatnosti i podataka}
\item{Zaštita intelektualne svojine}
\end{itemize}

\subsection{Elektronska trgovina}

Kako firme menjaju tradicionalne metode sa standardizovanim ra\-ču\-na\-rskim formama dolazi do potrebe da se nekako reguliše dogovor, transakcija ili nešto tome slično što nema papirni oblik sa potpisom u stvarnom životu, već je digitalno potpisano. Potrebno je utvrditi autentičnost dokumenta, kao i da li je transakcija validna. Kako je mnogo teže utvrdidi prepravljanje digitalnog dokumenta naspram fizičkog papirnog dokumenta na kome se svaka promena jasno vidi, potrebno je razviti način za kvalitetnu autentifikaciju i validaciju digitalnih dokumenata.\footnote{Univerzitet Princeton, Legal Issues and Information Security}

Elektronska trgovina se dosta razlikuje od tradicionalne vrste trgovine. Kada dodje do trasnakcije ko je nadležan? Ko je zadužen da održava zakon kod ove transkacije?

Na primer, ukoliko kupite televizor u obližnjoj prodavnici tehnike, tačno znate vaša prava. Ukoliko dođe do problema sa televizorom koji spada pod garanciju, prodavnica tehnike je dužna da Vam ispravi problem. U suprotnom garantni list može biti korišćen kao vrsta ugovora i ovo pitanje se može rešiti na sudu unutar zajednice. S druge strane ukoliko taj isti televizor kupite preko interneta, recimo sa drugog kraja sveta, iz Kine. Još interesantnije, ukoliko kupite to preko nekog veb servisa iz Norveške koji služi samo kao medijum između Vas i kineskog proizvođača, ne kao preprodavac, već samo kao usluga da stupite u kontakt sa proizvođačem televizora. Šta se tu dešava? Čiji zakoni se poštuju u slučaju problema? Kada se desi ovakva transakcija dosta kompleksnih stvari predstavljaju problem.

\subsection{Zaštita privatnosti i podataka}

Razvoj tehnologije je doprineo tome da se veliki broj informacija može sakupiti, obraditi, uporediti i iskoristiti za veoma kratak vremenski period. Samim tim došlo je do porasta potražnje ličnih, prihvatnih, podataka te su se otvorile i firme koje kao jedini cilj imaju prikupljanje podataka. Radi prebrzog razvoja informacionih tehnologija, postoje pravila u Evropi i Svetu koja regulišu tok i prikupljanje privatnih podataka. Problem je što se u različitim mestima poštuju različiti zakoni. Na primer ovi zakoni su se razlikovali u Sjedinjenim Američkim Državama i Evropskoj Uniji, što je moglo dovesti čak i do toga da se informacije nisu mogle razmenjivati između ovih dveju lokacija.

Jedna od najvećih organizacija zaduženih za zaštitu privatnosti i podataka je OECD\footnote{Organisation for Economic Co-operation and Development}. U pola država članica OECD-a će se uskoro uvesti zakoni ili su već uvedeni zakoni koji se drže pravila ove organizacije. U Austriji, Danskoj, Kanadi, Francuskoj, Nemačkoj, Luksemburgu, Norveškoj, Švetskoj i Sjedinjenim Američkim Državama je zakon usvojen, dok je u Belgiji, Islandu, Španiji, Švajcarskoj i Holandiji pripremljen predlog zakona.\footnote{OECD Guidelines on the Protection of Privacy and Transborder Flows of Personal Data}

Predloge zakona za OECD je napisala grupa eksperata predvođena sudijom Kirby\footnote{The Hon. Mr. Justice M.D. Kirby}, predsedavajućim članom australijske komisije za reformu zakona. Ovi predlozi su usvojeni i pušteni u delo 23. septembra 1980. godine.

Pravila OECD-a važe za sve privatne podatke, nebitno da li su u privatnom ili javnom sektoru, koji zbog načina procesiranja predstavljaju pretnju privatnosti. Takođe, ova pravila su napisana kao minimum koji bi svaka država trebala da poštuje, ali se očekuje i proširenje ovih pravila u okviru svake države kako bi se bolje prilagodila stanju u državi.

Neka od bitnijih pravila su:\footnote{OECD Guidelines on the Protection of Privacy and Transborder Flows of Personal Data}

\begin{itemize}
	\item{Treba postojati ograničenje u prikupljanju privatnih podataka i svako prikupljanje bi trebalo biti izvršeno legalno, takođe gde je to prikladno uz znanje ili odobrenje subjekta,}
	\item{Privatni podaci trebaju biti bitni za svrhe radi kojih se prikupljaju, i u te svrhe biti tačni i potpuni,}
	\item{Svrhe prikupljanja podataka trebaju biti rečena ne kasnije nego vreme prikupljanja podataka, i kasnija upotreba podataka ograničena za svrhe za koje su prikupljani,}
	\item{Privatni podaci trebaju biti zaštićeni prihvatljivim merama zaštite protiv rizika kao što su gubljenje ili neautorizovani pristup, uništenje, korišćenje, prepravljanje ili odavanje podataka.}
\end{itemize}

\subsection{Zaštita intelektualne svojine}

Intelektualna svojina je trenutno jedan od najvećih problema kod informacionih tehnologija. Zakoni koji regulišu zaštitu intelektualne sredine su jako abstraktni, što dozvoljava ljudima da zaobilaze ove zakone, kao i da se brane ukoliko su optuženi za kršenje ovih zakona. S obzirom da je masivnim porastom interneta na istom maltene sve dostupno, ljudi koristeći ove resurse ignorišu zakone, a zbog samog koncepta interneta i načina zaštite na internetu, ove ljude je jako teško pronaći, a kasnije i kazniti.

Najčešći problem na internetu je takozvani Copyright. Ovo daje kreatori originalnog dela ekskluzivna prava na to delo, uglavnom na određeno vreme. Pokriva veliku količinu kreativnih, intelektualnih ili artističkih formi. Ne odnosi se na ideje ili informacije, već na način na koji su izraženi\footnote{Wikipedia: Intellectual Property}





\appendix
	\addcontentsline{toc}{section}{Literatura}
	\bibliographystyle{plain}
	\bibliography{bibliografija}

\end{document}
