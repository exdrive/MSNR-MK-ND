%-----------------------------------------------------------------------------------------
%			Seminarski rad u okviru kursa Metodologija strucnog i naucnog rada 
%-----------------------------------------------------------------------------------------

\documentclass[a4paper]{article}
\usepackage{color}
\usepackage{url}
\usepackage[utf8]{inputenc}
\usepackage{multirow}
\usepackage[table,xcdraw]{xcolor}
\usepackage{graphicx}
\usepackage{adjustbox}
\usepackage[english,serbian]{babel}
\usepackage[unicode]{hyperref}
\hypersetup{colorlinks,citecolor=green,filecolor=green,linkcolor=blue,urlcolor=blue}
\newcommand\tab[1][1cm]{\hspace*{#1}}  
\newtheorem{primer}{Primer}[section]

\def\d{{\fontencoding{T1}\selectfont\dj}}
\def\D{{\fontencoding{T1}\selectfont\DJ}}

\begin{document}

\title{ Pravna i etička pitanja u IT  \\ \small{Seminarski rad u okviru kursa\\Metodologija stručnog i naučnog rada\\ Matematički fakultet}}

\author{Nikola Dokmanović, Miloš Krsmanović\\mi15117@alas.matf.bg.ac.rs, mi15263@alas.matf.bg.ac.rs}
\date{\today}
\maketitle 

\abstract{ [MOŽDA - PRERADITI] Šta je etika? Zašto je ona neophodna u svetu danas? Koliku ulogu igraju moralna pitanja u svetu informacionih tehonologija? Ko je odgovoran za etiku u pogledu, recimo, izrade softvera koji se koristi za prenos podataka o pacijentima između medicinskih ustanova? Samo su neka od pitanja kojima ćemo se pozabaviti i podstaći čitaoca na razmišljanje. 
Cilj ovog rada je pored informisanja čitaoca, podsticanje na razmišljanje i širenje ovih važnih pitanja i tema.


\section{Uvod}
\label{sec:Uvod}
	Nedovoljno upućeni ljudi često su mišljenja da u svetu informacionih tehnologija sve funkcioniše bez kolizija i nesuglasica. Kao i svaka druga grana privrede, tako se i IT suočavaju sa mnogim važnim pitanjima od kojih je jedno od značajnijih pitanje etike i morala. O etičkim pitanjim u svetu informacionih tehonologija nije lako govoriti, pa ipak, etičko ponašanje se očekuje od profesionalaca u gotovo svim granama industrije. Potrebno je, pre svega, definisati koje su to grane u IT-u koje zahtevaju bavljenje etikom. Skup svih grana informacionih tehnologija je znatno veći od drugih grana privrede i stoga je teško odrediti u  kojim granama je potrebno poštovati etički kodeks.

Kada je, na primer, život jedne osobe pogođen aspektima određenog softvera, pitanje sa kojima se IT zajednica suočava danas je da li ***dobavljači*** ili programeri treba da osećaju potrebu za etičkim kodeksom.
Kako se svet tehnologije razvija i napreduje veoma brzo, izrazito je teško sprovesti zakone i doneti ispravne odluke po pitanju morala i etike u svetu informaiconih tehnologija. Bez obzira na to, jasno da stanje etike u IT-u treba da se poboljša.

\section{Šta je etika?}

Prvenstveno, etika je kodeks ponašanja i profesionalne organizacije koji poštuju (prate) institucije, ustanove i stručnjaci u svim zanimanjima kako bi se osigurala privatnost i poverenje korisnika. Etika ne uklanja odgovornost sa pojedinca i prenosi je na ustanovu, naprotiv, za etiku je neophodno razumevanje koje dolazi iznutra. Često se navodi da je etika \emph{“poznavanje razlike između onoga što imamo pravo da radimo i onoga što bismo trebali da radimo”.}\footnote{Poter Stjuart, sudija Vrhovnog suda SAD od 1958-1981}

Primer može biti poznata relacija lekar-pacijent gde je poverljivost podrazumevani standard koji štiti privatnost medicinskih informacija.

\section{Etička pitanja za IT stručnjake}

Lekari, advokati i drugi profesionalci čiji posao utiče na živote drugih ljudi obično pohađaju, kao deo njihove formalne obuke, kurseve koji se bave etičkim pitanjima u njihovoj profesiji. To nije slučaj sa IT stručnjacima, a upravo oni često imaju pristup tajnim podacima i saznanjima o pojedincima i kompanijama. To je \emph{moć} koja se može zloupotrebiti, namerno ili nenamerno. Iako ne postoje standardizovane obuke za IT stručnjake, sve više kompanija koje proizvode softver počinje da obraća pažnju na etiku.

Profesionalci u mnogim strukama se često susreću sa pitanjima etike a da toga nisu ni svesni. Kao primer možemo navesti sledeću situaciju: Prelistavanjem slučajnih dokumenata otkrili ste važne službene tajne. Šta ako napustite tu kompaniju i zaposlite se kod konkurenata? Da li je pogrešno koristiti to znanje na novom poslu? Da li bi još gore bilo da ta dokumenta odštampate i ponesete ih sa sobom?
Moguć je, recimo, i ovakav scenario: Dokumenti koje ste čitali potvrđuju da kompanija krši vladine propise ili zakone. Da li će u vama prevladati moralna odgovornost da te nepravilnosti prijavite ili etička obaveza da poštujete privatnost svog poslodavca.
Informatičkim stručnjacima trebalo bi omogućiti jasne smernice koje bi im pomogle u rešavanju etičkih pitanja.

\subsection{Zašto su etičke smernice potrebne}

Za razliku od starijih, davno ustanovljenih profesija kao što su medicina i pravo, većina moralnih i etičkih pitanja sa kojima se susreću stručnjaci u IT-u ne nalaze se ni u jednom zakonu ili propisu. Ne postoji standard koji uspostavlja nadzorno telo, kao što je slučaj sa medicinom i pravima gde su ustanovljena nacionalna medicinska udruženja i advokatske komore koje prate detaljan etički kodeks. 
Međutim, pitanje etičkog ponašanja u informacionim tehnologijama počinje da se rešava. Dobrovoljna, neprofitna udrženja kao što je ACM\footnote{ACM, Američka Asocijacija za kompjuterske sisteme osnovana 1947. godine} razvijaju svoje etičke kodekse profesionalnog ponašanja, koji mogu poslužiti kao smernica za pojedince i druge organizacije.

\begin{table}[]
\centering
\begin{adjustbox}{center, width=\textwidth}
\begin{tabular}{|l|l|c|c|c|}
\hline
\multicolumn{2}{|c|}{\multirow{2}{*}{\textbf{Grupa zaposlenih koji su:}}}         & \multicolumn{3}{c|}{\textbf{Godina anketiranja}} \\ \cline{3-5} 
\multicolumn{2}{|c|}{}  & \textbf{2007.}    & \textbf{2008.}   & \textbf{2009.}   \\ \hline
\multicolumn{2}{|l|}{\textbf{Prisustvovali kršenju etičkih pravila}}      & 45\%  & 49\% & 63\%  \\ \hline
\multicolumn{2}{|l|}{\textbf{Bili primorani na kršenje moralnih pravila}} & 58\%  & 63\% & 65\%  \\ \hline
\multicolumn{2}{|l|}{\textbf{Primetili manjak etike u radnom okruženju}}  & 39\%  & 35\% & 42\%  \\ \hline

\end{tabular}
\end{adjustbox}
\caption{Rezultati ankete sprovedene u državnim preduzećima}    \footnote{Napomena: Anketirani su zaposleni u preduzećima u Sjedinjenim Američkim Državama}
\label{}
\end{table}

\section{Osnovna pravna pitanja vezana za IT}

Kako se tehnologija razvija potrebno je da je prate zakoni koji obezbeđuju korišćenje tehnologije. Razvijanjem tehnologije konstantno je potrebno unapređivati zakone vezane za informacijone tehnologije, ali zbog jako brze evolucije zakon jednostavno ne stiže da se unapredi dovoljnom brzinom.

Zakoni koji danas obezbeđuju trgovinu, privatnost podataka i intelektualnu svojinu na primer, pisani su za vremena kada se trgovalo sa listom papira, kada se komuniciralo preko telegrafa i kada su se dokumenti kucali na kucaćim mašinama.\footnote{Univerzitet Princeton, Legal Issues and Information Security} Sve ovo se dramatično razvilo tokom godina jačanjem informacionih tehnologija.

Obradićemo tri oblasti gde konstantna promena informacionih tehnologija drastično utiče na zakon:

\begin{itemize}
\item{Elektronska trgovina}
\item{Zaštita privatnosti i podataka}
\item{Zaštita intelektualne svojine u administraciji digitalnih biblioteka}
\end{itemize}

\subsection{Elektronska trgovina}

Kako firme menjaju tradicionalne metode sa standardizovanim računarskim formama dolazi do potrebe da se nekako reguliše dogovor, transakcija ili nešto tome slično što nema papirni oblik sa potpisom u stvarnom životu, već je digitalno potpisano. Potrebno je utvrditi autentičnost dokumenta, kao i da li je transakcija validna. Kako je mnogo teže utvrdidi prepravljanje digitalnog dokumenta naspram fizičkog papirnog dokumenta na kome se svaka promena jasno vidi, potrebno je razviti način za kvalitetnu autentifikaciju i validaciju digitalnih dokumenata.\footnote{Univerzitet Princeton, Legal Issues and Information Security}

Elektronska trgovina se dosta razlikuje od tradicionalne vrste trgovine. Kada dodje do trasnakcije ko je nadležan? Ko je zadužen da održava zakon kod ove transkacije?

Na primer, ukoliko kupite televizor u obližnjoj prodavnici tehnike, tačno znate vaša prava. Ukoliko dođe do problema sa televizorom koji spada pod garanciju, prodavnica tehnike je dužna da Vam ispravi problem. U suprotnom garantni list može biti korišćen kao vrsta ugovora i ovo pitanje se može rešiti na sudu unutar zajednice. S druge strane ukoliko taj isti televizor kupite preko interneta, recimo sa drugog kraja sveta, iz Kine. Još interesantnije, ukoliko kupite to preko nekog veb servisa iz Norveške koji služi samo kao medijum između Vas i kineskog proizvođača, ne kao preprodavac, već samo kao usluga da stupite u kontakt sa proizvođačem televizora. Šta se tu dešava? Čiji zakoni se poštuju u slučaju problema? Kada se desi ovakva transakcija dosta kompleksnih stvari predstavljaju problem.

\addcontentsline{toc}{section}{Literatura}
\appendix
	\addcontentsline{toc}{section}{Literatura}
	\bibliographystyle{plain}
	\bibliography{bibliografija}

\end{document}