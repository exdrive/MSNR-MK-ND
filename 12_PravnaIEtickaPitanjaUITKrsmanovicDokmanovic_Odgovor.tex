% !TEX encoding = UTF-8 Unicode
\documentclass[a4paper]{report}

\usepackage[T2A]{fontenc} % enable Cyrillic fonts
\usepackage[utf8x,utf8]{inputenc} % make weird characters work
\usepackage[serbian]{babel}
%\usepackage[english,serbianc]{babel}
\usepackage{amssymb}

\usepackage{color}
\usepackage{url}
\usepackage[unicode]{hyperref}
\hypersetup{colorlinks,citecolor=green,filecolor=green,linkcolor=blue,urlcolor=blue}

\newcommand{\odgovor}[1]{\textcolor{blue}{#1}}

\begin{document}

\title{Dopunite naslov svoga rada\\ \small{Nikola Dokmanović, Miloš Krsmanović}}

\maketitle

\tableofcontents

\chapter{Uputstva}
\emph{Prilikom predavanja odgovora na recenziju, obrišite ovo poglavlje.}

Neophodno je odgovoriti na sve zamerke koje su navedene u okviru recenzija. Svaki odgovor pišete u okviru okruženja \verb"\odgovor", \odgovor{kako bi vaši odgovori bili lakše uočljivi.} 
\begin{enumerate}

\item Odgovor treba da sadrži na koji način ste izmenili rad da bi adresirali problem koji je recenzent naveo. Na primer, to može biti neka dodata rečenica ili dodat pasus. Ukoliko je u pitanju kraći tekst onda ga možete navesti direktno u ovom dokumentu, ukoliko je u pitanju duži tekst, onda navedete samo na kojoj strani i gde tačno se taj novi tekst nalazi. Ukoliko je izmenjeno ime nekog poglavlja, navedite na koji način je izmenjeno, i slično, u zavisnosti od izmena koje ste napravili. 

\item Ukoliko ništa niste izmenili povodom neke zamerke, detaljno obrazložite zašto zahtev recenzenta nije uvažen.

\item Ukoliko ste napravili i neke izmene koje recenzenti nisu tražili, njih navedite u poslednjem poglavlju tj u poglavlju Dodatne izmene.
\end{enumerate}

Za svakog recenzenta dodajte ocenu od 1 do 5 koja označava koliko vam je recenzija bila korisna, odnosno koliko vam je pomogla da unapredite rad. Ocena 1 označava da vam recenzija nije bila korisna, ocena 5 označava da vam je recenzija bila veoma korisna. 

NAPOMENA: Recenzije ce biti ocenjene nezavisno od vaših ocena. Na osnovu recenzije ja znam da li je ona korisna ili ne, pa na taj način vama idu negativni poeni ukoliko kažete da je korisno nešto što nije korisno. Vašim kolegama šteti da kažete da im je recenzija korisna jer će misliti da su je dobro uradili, iako to zapravo nisu. Isto važi i na drugu stranu, tj nemojte reći da nije korisno ono što jeste korisno. Prema tome, trudite se da budete objektivni. 
\chapter{Recenzent \odgovor{--- ocena: 4} }


\section{O čemu rad govori?}
% Напишете један кратак пасус у којим ћете својим речима препричати суштину рада (и тиме показати да сте рад пажљиво прочитали и разумели). Обим од 200 до 400 карактера.
U ovom radu nas autori na početku upoznaju sa pojmom etike, a zatim objašnjavaju pojavu i uticaj etičkih pitanja u informacionim tehnologijama. Takođe se 
osvrću i na to da ne postoje precizni etički kodeksi, ali da potreba za preciznim etičkim smernicama postaje sve potrebnija. Na kraju se govori i o pravnim
pitanjima kao što su zaštita intelektualne svojine i zaštita podataka kao i o elektronskoj trgovini i o strahu ljudi da ne budu prevareni kada kupuju preko interneta.

\section{Krupne primedbe i sugestije}
% Напишете своја запажања и конструктивне идеје шта у раду недостаје и шта би требало да се промени-измени-дода-одузме да би рад био квалитетнији.
Nemam krupnijih primedbi.

\section{Sitne primedbe}
% Напишете своја запажања на тему штампарских-стилских-језичких грешки
Čitajuću tekst primetio sam par štamparskih i jezičkih grešaka:
\begin{itemize}
\item U poslednjoj rečenici uvoda piše jasno umesto jasno je
\item Na početku poslednjeg pasusa na drugoj strani piše nebrojeno mnogo gde je nebrojeno možda višak
\item U drugoj rečenici trećeg poglavlja piše informacijone umesto informacione
\item U poslednjoj rečenici na četvrtoj strani piše utvrdidi umesto utvrditi
\item Na petoj strani nejasan je smisao rečenice: S druge strane ukoliko taj
isti televizor kupite preko interneta, recimo sa drugog kraja sveta, iz Kine.
\item U poglavlju 3.2. piše Svetu, a pravilno je svetu
\item U poglavlju 3.2. fali zarez posle Na primer
\item U zaključku u drugom pasusu piše idalje umesto i dalje
\end{itemize}

\odgovor{Sve navedene primedbe se odnose na slovne i jezičke greške. Primedbe su prihvaćene i zamene su izvršene.}

\section{Provera sadržajnosti i forme seminarskog rada}
% Oдговорите на следећа питања --- уз сваки одговор дати и образложење

\begin{enumerate}
\item Da li rad dobro odgovara na zadatu temu?\\
Rad odgovara na zadatu temu.
\item Da li je nešto važno propušteno?
\item Da li ima suštinskih grešaka i propusta?\\
Nisam primetio da je nešto važno propušteno ili da ima nekih suštinskih grešaka.
\item Da li je naslov rada dobro izabran?\\
Možda bi bolji naslov rada bio pravna i etička pitanja nego pravne i etičke obaveze.
\item Da li sažetak sadrži prave podatke o radu?\\
Sažetak sadrži prave podatke o radu.
\item Da li je rad lak-težak za čitanje?\\
Meni je rad uglavnom bio lak za čitanje.
\item Da li je za razumevanje teksta potrebno predznanje i u kolikoj meri?\\
Ne mislim da je potrebno neko specifično predznanje.
\item Da li je u radu navedena odgovarajuća literatura?
\item Da li su u radu reference korektno navedene?\\
I literatura i reference su korektno navedene.
\item Da li je struktura rada adekvatna?\\
Adekvatna je struktura rada.
\item Da li rad sadrži sve elemente propisane uslovom seminarskog rada (slike, tabele, broj strana...)?\\
Ispunjeni su svi uslovi seminarskog rada.
\item Da li su slike i tabele funkcionalne i adekvatne?\\
Slike i tabele su adekvatne.
\end{enumerate}

\odgovor{Naslov rada nije promenjen u skladu sa zahtevima recenzenta, jer je naslov koji je predložio recenzent zapravo dodeljena tema pa to ne bi bilo odgovarajuće.}

\section{Ocenite sebe}
% Napišite koliko ste upućeni u oblast koju recenzirate: 
% a) ekspert u datoj oblasti
% b) veoma upućeni u oblast
% c) srednje upućeni
% d) malo upućeni 
% e) skoro neupućeni
% f) potpuno neupućeni
% Obrazložite svoju odluku
Smatram da sam skoro neupućen u ovu oblast s obzirom da se nisam previše interesovao nikada za ovu oblast.

\chapter{Recenzent \odgovor{--- ocena:4} }


\section{O čemu rad govori?}
% Напишете један кратак пасус у којим ћете својим речима препричати суштину рада (и тиме показати да сте рад пажљиво прочитали и разумели). Обим од 200 до 400 карактера.
Rad definiše pojam etike u svetu informacionih tehnologija. Razmatra se zašto je pitanje etičkog ponašanja u informacionim tehnologijama jako bitno, i zašto je bitno unapređivati zakone vezane za informacione tehnologije. Opisani su problemi koji se javljaju u pokušaju da se unaprede takvi zakoni. Obrađene su tri oblasti gde konstantna promena informacionih tehnologija drastično utiče na zakon.

\section{Krupne primedbe i sugestije}
% Напишете своја запажања и конструктивне идеје шта у раду недостаје и шта би требало да се промени-измени-дода-одузме да би рад био квалитетнији.
Ne postoje krupnije primedbe i sugestije.

\section{Sitne primedbe}
% Напишете своја запажања на тему штампарских-стилских-језичких грешки
U nastavku su navedene predložene izmene delova teksta koji sadrže greške. Reči koje je potrebno ispraviti su podebljane.
\begin{enumerate}
	\item Uvod, prvi pasus, treća rečenica: \textit{O etičkim \textbf{pitanjim} u svetu informacionih tehonologija nije lako govoriti, pa ipak, etičko ponašanje se očekuje od profesionalaca u gotovo svim granama industrije.}
	\item Uvod, drugi pasus, druga rečenica: \textit{Pravni odgovor na ovo pitanje ne postoji, jer kako se svet tehnologije razvija i napreduje  veoma brzo, izrazito je teško sprovesti zakone i doneti ispravne odluke po pitanju morala i etike u svetu \textbf{informaiconih} tehnologija.}
	\item Osnovna pravna pitanja vezana za IT, prvi pasus, druga rečenica: \textit{Razvijanjem tehnologije konstantno je potrebno unapređivati zakone vezane za \textbf{informacijone} tehnologije, ali zbog jako brze evolucije zakon jednostavno ne stiže da se unapredi dovoljnom brzinom.}
	\item Elektronska trgovina, prvi pasus, treća rečenica: \textit{Kako je mnogo teže \textbf{utvrdidi} prepravljanje digitalnog dokumenta naspram fizičkog papirnog dokumenta na kome se svaka promena jasno vidi, potrebno je razviti način za kvalitetnu autentifikaciju i validaciju digitalnih dokumenata.}
	\item Elektronska trgovina, drugi pasus, druga i treća rečenica: \textit{Kada dodje do \textbf{trasnakcije} ko je nadležan? Ko je zadužen da održava zakon kod ove transkacije?}
	\item Zaštita privatnosti i podataka, prvi pasus, druga rečenica: \textit{Samim tim došlo je do porasta potražnje ličnih, \textbf{prihvatnih}, podataka te su se otvorile i firme koje kao jedini cilj imaju prikupljanje podataka.}
	\item Zaštita privatnosti i podataka, drugi pasus, treća rečenica: \textit{U Austriji, Danskoj, Kanadi, Francuskoj, Nemačkoj, Luksemburgu, Norveškoj, \textbf{Šve\-tskoj} i Sjedinjenim Američkim Državama je zakon usvojen, dok je u Belgiji, Islandu, Španiji, Švajcarskoj i Holandiji pripremljen predlog zakona.}
\end{enumerate}
\odgovor{Svi predlozi za ispravke su prihvaćeni i zamene su izvršene. Zahvaljujemo se recenzentu na otkrivanju slovnih grešaka koje su oku često promicale, čak i nakon ponovnog čitanja teksta. }
\section{Provera sadržajnosti i forme seminarskog rada}
% Oдговорите на следећа питања --- уз сваки одговор дати и образложење

\begin{enumerate}
\item Da li rad dobro odgovara na zadatu temu?\\
Ovaj rad odgovara na sva pitanja predviđena ovom temom.
\item Da li je nešto važno propušteno?\\
Nema važnijih stvari koje su propuštene.
\item Da li ima suštinskih grešaka i propusta?\\
Rad sadrži greške nastale prilikom kucanja, koje su naznačene u prethodnoj sekciji.
\item Da li je naslov rada dobro izabran?\\
Naslov rada je u skladu sa sadržajem rada.
\item Da li sažetak sadrži prave podatke o radu?\\
Sažetak sadrži prave podatke o radu.
\item Da li je rad lak-težak za čitanje?\\
Rad je lak za čitanje, i pruženi su primeri koji dodatno olakšavaju razumevanje obrađenih tema.
\item Da li je za razumevanje teksta potrebno predznanje i u kolikoj meri?\\
Za razumevanje teksta nije potrebno predznanje.
\item Da li je u radu navedena odgovarajuća literatura?\\
U radu je navedena odgovarajuća literatura, i citirana je u okviru rada.
\item Da li su u radu reference korektno navedene?\\
Reference su korektno navedene.
\item Da li je struktura rada adekvatna?\\
Rad sadrži naslov, sažetak, sadržaj, uvod, razradu, zaključak i literaturu. Struktura rada je adekvatna.
\item Da li rad sadrži sve elemente propisane uslovom seminarskog rada (slike, tabele, broj strana...)?\\
Rad sadrži slike, tabele, i ograničenja koja se odnose na broj strana i literaturu su zadovoljena.
\item Da li su slike i tabele funkcionalne i adekvatne?\\
Slike i tabele su funkcionalne i adekvatne, i referisane su u okviru teksta.
\end{enumerate}

\section{Ocenite sebe}
% Napišite koliko ste upućeni u oblast koju recenzirate: 
% a) ekspert u datoj oblasti
% b) veoma upućeni u oblast
% c) srednje upućeni
% d) malo upućeni 
% e) skoro neupućeni
% f) potpuno neupućeni
% Obrazložite svoju odluku
Srednje sam upućen u oblast koju recenziram. Sa ovom temom sam se susretao na nekim od prethodnih kurseva.

\chapter{Recenzent \odgovor{--- ocena:4} }

\section{O čemu rad govori?}
% Напишете један кратак пасус у којим ћете својим речима препричати суштину рада (и тиме показати да сте рад пажљиво прочитали и разумели). Обим од 200 до 400 карактера.
Rad objašnjava šta je etika, naglašava značaj etike u IT sektoru. Daje sugestije kako treba postupati u skladu sa etikom kao IT stručnjak, ali i svakodnevni korisnik računara, i objašnjava značaj obrazovanja IT stručnjaka o etici.\\
Potom je objašnjen značaj donošenja zakona vezanih za IT sektor (internet kupovina, privatnost podataka, intelektualna svojina), ali i koje probleme donosi neredovno ažuriranje istih zakona ili njihovo potpuno odsustvo.

\section{Krupne primedbe i sugestije}
% Напишете своја запажања и конструктивне идеје шта у раду недостаје и шта би требало да се промени-измени-дода-одузме да би рад био квалитетнији.
Naslov mi se čini predugačkim. Možda bi bilo bolje da glasi: $"$Pravne i etičke obaveze u IT svetu$"$. Verujem da ljudi lakše prepoznaju $"$IT$"$ kao oznaku za svet računarstva nego $"$informacione tehnologije$"$.  \\

\odgovor{Sa ovom primedbom se u potpunosti slažemo. Naslov je promenjen u \textbf{Pravne i \-etičke obaveze u IT svetu}}

\section{Sitne primedbe}
% Напишете своја запажања на тему штампарских-стилских-језичких грешки
Nemam nikakvih sitnih primedbi primedbi.

\section{Provera sadržajnosti i forme seminarskog rada}
% Oдговорите на следећа питања --- уз сваки одговор дати и образложење

\begin{enumerate}
\item Da li rad dobro odgovara na zadatu temu?\\ Rad odgovara na sva pitanja i to veoma dobro.
\item Da li je nešto važno propušteno?\\ Ne bih rekao da je išta propušteno.
\item Da li ima suštinskih grešaka i propusta?\\ Suštinskih grešaka nema. Tema nije promašena, rad ima suštinu.
\item Da li je naslov rada dobro izabran?\\ Naslov mi se čini predugačkim.
\item Da li sažetak sadrži prave podatke o radu?\\ Sažetak je lep i fino uvodi u priču. Daje veoma lep uvid u to šta se nalazi u radu.
\item Da li je rad lak-težak za čitanje?\\ Rad se veoma lako čita, i veoma je zanimljiv.
\item Da li je za razumevanje teksta potrebno predznanje i u kolikoj meri?\\ Nije potrebno nikakvo pravno znanje, niti stručno znanje iz oblasti IT.
\item Da li je u radu navedena odgovarajuća literatura?\\ Literatura je odgovarajuća.
\item Da li su u radu reference korektno navedene?\\ Reference su korektno navedene.
\item Da li je struktura rada adekvatna?\\ Ima sažetak, fin uvod, odličnu razradu i odličan zaključak.
\item Da li rad sadrži sve elemente propisane uslovom seminarskog rada (slike, tabele, broj strana...)?\\ Rad sadrži sve elemente propisane uslovom seminarskog rada.
\item Da li su slike i tabele funkcionalne i adekvatne?\\ Slike i tabele su funkcionalne i adekvatne.
\end{enumerate}

\section{Ocenite sebe}
% Napišite koliko ste upućeni u oblast koju recenzirate: 
% a) ekspert u datoj oblasti
% b) veoma upućeni u oblast
% c) srednje upućeni
% d) malo upućeni 
% e) skoro neupućeni
% f) potpuno neupućeni
% Obrazložite svoju odluku
b) Veoma sam upućen u oblast, pogotovo o privatnosti podataka jer se bavim promocijom iste u okviru organizacije Mozilla Srbija.

\chapter{Recenzent \odgovor{--- ocena: 5} }


\section{O čemu rad govori?}
% Напишете један кратак пасус у којим ћете својим речима препричати суштину рада (и тиме показати да сте рад пажљиво прочитали и разумели). Обим од 200 до 400 карактера.

U radu ''Pravne i etičke obaveze u svetu'' navedeni su i prodiskutovani neki od važnijih etičkih i pravnih pitanja u svetu informacionih tehnologija. Etička pitanja predstavljaju sivu zonu u mnogim oblastima, ali dok u drugim sferama neka etička pravila postoje i jasno su definisana, u informacionim tehnologijama i dalje ne postoji jedinstven skup pravila. Dodatno, svest o potrebi za etičkim kodeksom još uvek nije na nivou. Pravna pitanja u oblasti informacionih tehnologija su takođe problematična zbog teškog praćenja aktivnosti po internetu i prirode podataka.

\section{Krupne primedbe i sugestije}
% Напишете своја запажања и конструктивне идеје шта у раду недостаје и шта би требало да се промени-измени-дода-одузме да би рад био квалитетнији.
Rad je dobro struktuiran i napisan. Jedino bi mogli da se navedu neki primeri etičkh pitanja koji su specifični samo u oblasti informacionih tehnologija. \\

\odgovor{Navođenje još jednog primera koji bi se direktno ticao informacionih tehnologija je izostavljeno sa namerom. Naime, primeri koji su navedeni u poglavlju 2.2 su lišeni svih obeležja i pojedinosti određenog posla kako bi mogli predstavljati primer za etičke dileme u IT svetu.}

\section{Sitne primedbe}
% Напишете своја запажања на тему штампарских-стилских-језичких грешки

U poglavlju 3, prvi pasus: informacijone tehnologije --> informacione tehnologije, ...jako brze evolucije --> brze ili veoma brze evolucije

Citati bi trebalo da se navode pre tačke, odnosno ovako citirati [1]. A ne ovako posle tačke.[1] 

U poglavlju 3.1, drugi pasus: \textbf{dodje} do trasnakcije

U poglavlju 3.2, prvi pasus: postoje pravila u Evropi i \textbf{Svetu} koja... --> svetu 

U poglavlju 3.3, prvi pasus: na istom \textbf{maltene} sve dostupno --> gotovo sve ili skoro sve

\odgovor{Sve primedbe recenzenta su prihvaćene, sve slovne greške su ispravljene. Takođe, i način citiranja je promenjen tako da se sada citira pre interpunkcijskih znakova.}

\section{Provera sadržajnosti i forme seminarskog rada}
% Oдговорите на следећа питања --- уз сваки одговор дати и образложење

\begin{enumerate}
\item Da li rad dobro odgovara na zadatu temu?\\

\textbf{Da}. 

\item Da li je nešto važno propušteno?\\

\textbf{Ne.} Tema je definitivno vrlo široka i moguće je napisati dosta, ali u ovom radu su izdvojeni neki od bitnijih problema.

\item Da li ima suštinskih grešaka i propusta?\\

\textbf{Ne.} Po mom mišljenju nema.

\item Da li je naslov rada dobro izabran?\\

\textbf{Da.} Naslov daje informaciju o temi koju rad obrađuje.

\item Da li sažetak sadrži prave podatke o radu?\\

\textbf{Da.} Abstrakt pokriva opisano u radu.

\item Da li je rad lak-težak za čitanje?\\

\textbf{Lak.} Rad je stilski i strukturno dobro organizovan što ga čini lakim za čitanje.

\item Da li je za razumevanje teksta potrebno predznanje i u kolikoj meri?\\

\textbf{Ne.} Rad se bavi opštom temom koja je deo svih oblasti, a specifičnosti problematike u IT su dobro objašnjenje.

\item Da li je u radu navedena odgovarajuća literatura?\\

\textbf{Da.}

\item Da li su u radu reference korektno navedene?\\

\textbf{Da.}

\item Da li je struktura rada adekvatna?\\

\textbf{Da.}

\item Da li rad sadrži sve elemente propisane uslovom seminarskog rada (slike, tabele, broj strana...)?\\

\textbf{Da.} Obim rada, broj referenci, slika i tabela zadovoljava postavljen minimum. 

\item Da li su slike i tabele funkcionalne i adekvatne?\\

\textbf{Da.} Tabelarno su prikazani rezultati ankete koja je relevantna za rad i ovakav način prikazivanja značajno doprinosi uviđanju zakonitosti u rezultatima. Slika navedena u radu je funkcionalna i adekvatna.
\end{enumerate}

\section{Ocenite sebe}
% Napišite koliko ste upućeni u oblast koju recenzirate: 
% a) ekspert u datoj oblasti
% b) veoma upućeni u oblast
% c) srednje upućeni
d) malo upućeni

Tema je generalno poznata, a sa specifičnostima problema u IT sam se susrela prilikom istraživanja moje teme za seminarski.
% e) skoro neupućeni
% f) potpuno neupućeni
% Obrazložite svoju odluku

\chapter{Dodatne izmene}
%Ovde navedite ukoliko ima izmena koje ste uradili a koje vam recenzenti nisu tražili. 
\end{document}