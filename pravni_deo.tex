\documentclass{article}

\usepackage[utf8]{inputenc}

\begin{document}

\section{Osnovna pravna pitanja vezana za IT}

Kako se tehnologija razvija potrebno je da je prate zakoni koji obezbeđuju korišćenje tehnologije. Razvijanjem tehnologije konstantno je potrebno unapređivati zakone vezane za informacijone tehnologije, ali zbog jako brze evolucije zakon jednostavno ne stiže da se unapredi dovoljnom brzinom.

Zakoni koji danas obezbeđuju trgovinu, privatnost podataka i intelektualnu svojinu na primer, pisani su za vremena kada se trgovalo sa listom papira, kada se komuniciralo preko telegrafa i kada su se dokumenti kucali na kucaćim mašinama.\footnote{Univerzitet Princeton, Legal Issues and Information Security} Sve ovo se dramatično razvilo tokom godina jačanjem informacionih tehnologija.

Obradićemo tri oblasti gde konstantna promena informacionih tehnologija drastično utiče na zakon:

\begin{itemize}
\item{Elektronska trgovina}
\item{Zaštita privatnosti i podataka}
\item{Zaštita intelektualne svojine u administraciji digitalnih biblioteka}

\section{Elektronska trgovina}

Kako firme menjaju tradicionalne metode sa standardizovanim računarskim formama dolazi do potrebe da se nekako reguliše dogovor, transakcija ili nešto tome slično što nema papirni oblik sa potpisom u stvarnom životu, već je digitalno potpisano. Potrebno je utvrditi autentičnost dokumenta, kao i da li je transakcija validna. Kako je mnogo teže utvrdidi prepravljanje digitalnog dokumenta naspram fizičkog papirnog dokumenta na kome se svaka promena jasno vidi, potrebno je razviti način za kvalitetnu autentifikaciju i validaciju digitalnih dokumenata.\footnote{Univerzitet Princeton, Legal Issues and Information Security}

Elektronska trgovina se dosta razlikuje od tradicionalne vrste trgovine. Kada dodje do trasnakcije ko je nadležan? Ko je zadužen da održava zakon kod ove transkacije?

Na primer, ukoliko kupite televizor u obližnjoj prodavnici tehnike, tačno znate vaša prava. Ukoliko dođe do problema sa televizorom koji spada pod garanciju, prodavnica tehnike je dužna da Vam ispravi problem. U suprotnom garantni list može biti korišćen kao vrsta ugovora i ovo pitanje se može rešiti na sudu unutar zajednice. S druge strane ukoliko taj isti televizor kupite preko interneta, recimo sa drugog kraja sveta, iz Kine. Još interesantnije, ukoliko kupite to preko nekog veb servisa iz Norveške koji služi samo kao medijum između Vas i kineskog proizvođača, ne kao preprodavac, već samo kao usluga da stupite u kontakt sa proizvođačem televizora. Šta se tu dešava? Čiji zakoni se poštuju u slučaju problema? Kada se desi ovakva transakcija dosta kompleksnih stvari predstavljaju problem.




\end{document}